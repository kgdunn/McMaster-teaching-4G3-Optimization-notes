%!TEX root = ../4G3-course.tex

\begin{frame}\frametitle{Credits}
	\begin{exampleblock}{}
		\centering {\color{myOrange}{Dr. Thomas Marlin}}, {\color{myBlue}{Dr. Benoit Chachuat}}, {\color{myGreen}{Dr. San Yip}}
	\end{exampleblock}
	\begin{itemize}
		\item	Have been the main instructors in the past decade
		\item	Course outline and topics covered are similar to theirs
		\item	We will use their notes, slides, and other materials for most of the course
	\end{itemize}
\end{frame}

\begin{frame}\frametitle{Background}
	{\color{myGreen}{About myself}}
	\begin{itemize}
		\item	Undergraduate degree from University of Cape Town, 1999
		\item	Masters degree from McMaster, 2002 (not a ``doctor'', just call me by my first name please)
		\item	Worked with a number of companies from 2002 to 2011 on data analysis and consulting projects
		\item	Masters thesis was focused entirely on using optimization tools
		\item	Now working full-time at McMaster since July 2012
		\item	Office is in BSB, room B105
		\item	Arrange a meeting: \url{kevin.dunn@mcmaster.ca}
			\begin{itemize}
				\item	I will meet as soon as I am available
				\item	but you must email me with a time, and a short description of what you want to talk about
			\end{itemize}
		\item	Cell: (905) 921 5803
	\end{itemize}
\end{frame}

\begin{frame}\frametitle{Teaching assistants}
	\begin{columns}[c]
		\column{0.85\textwidth}
			{\color{myGreen}{James Scott}}
			\begin{itemize}
				\item	\url{che4g3@gmail.com}
				\item	JHE, room 132
				\item	Currently doing his Masters with Thomas Adams
			\end{itemize}
			
			\vspace{0.8cm}
			{\color{myGreen}{Jaffer Ghouse}}
			\begin{itemize}
				\item	\url{che4g3@gmail.com}
				\item	JHE, room 370
				\item	Currently doing his Ph.D with Thomas Adams
			\end{itemize}
			
		\column{0.20\textwidth}
			\centerline{\includegraphics[width=\textwidth]{\imagedir/teaching/TA-photos/James-Scott.jpg}}
			
			\vspace{0.5cm}
			\centerline{\includegraphics[width=\textwidth]{\imagedir/teaching/TA-photos/jaffer-ghouse.jpg}}

	\end{columns}
	
	\vspace{0.3cm}
	{\color{myOrange}\scriptsize{Office hours: you must arrange to meet with TAs by email, when mutually convenient}}
\end{frame}

\begin{frame}\frametitle{Instructor, TA and student relationship}
	\begin{itemize}
		\item	You can expect TAs and I to answer emails promptly

		\vspace{12pt}
		\item	If you have questions
			\begin{enumerate}
				\item	Please email the TA with CC to me \hfill {\tiny{\color{myOrange}{$\longleftarrow$ hopefully this solves your problem}}}
				\item	if not, set up meeting with TA or myself
			\end{enumerate}
		\item	Please email from your McMaster address (filtering)
	\end{itemize}
\end{frame}

\begin{frame}\frametitle{Video and audio recordings}
	\begin{itemize}
		\item	As long as feasible, I will try to video record all classes
		\item	Useful if you miss a class
		\item	Quality might not be the best
		\item	Usually available 24 to 48 hours after the class
		\item	Audio recordings will also be made available, when possible
	\end{itemize}
\end{frame}

\begin{frame}\frametitle{References and readings}

	Notes will be made available via the course website periodically.
	
	Please print them and bring to class.


	\vspace{12pt}

	\begin{columns}[t]
		\column{0.80\textwidth}
			\emph{Recommended}: Rardin, ``Optimization in Operations Research'' 
		\column{0.10\textwidth}
			\vspace{-1cm}
			\begin{center}
				\includegraphics[width=1.5\textwidth]{\imagedir/teaching/textbook-covers/Rardin-Optimization.jpeg}
			\end{center}
	\end{columns}

	\vspace{12pt}
	There are many other textbooks and references listed on the course website; many of them are free.
\end{frame}

\begin{frame}\frametitle{Course website}
	\begin{exampleblock}{}
		\centering
		\href{http://learnche.mcmaster.ca/4G3}{http://learnche.mcmaster.ca/4G3}
	\end{exampleblock}
	
	\vspace{1cm}
	\begin{itemize}
		\item	Check \textbf{several times per week} for announcements {\tiny (top left)}
		\begin{flushright}
			\includegraphics[width=0.5\textwidth]{\imagedir/teaching/website-snapshots/4G3-2015-course-website.png}
		\end{flushright}

		\vspace{0cm}
		\item	Follow the Twitter feed: \href{https://twitter.com/opt4eng}{@opt4eng}
		\item	Check for updated slides and notes
		\item	Tutorials and assignments will be posted here
	\end{itemize}
\end{frame}

\begin{frame}\frametitle{Course feedback via Learning website}
	\begin{itemize}
		\item	I might not have explained something clearly;
		\item	you didn't get a chance to ask a question, \emph{etc}
	\end{itemize}
	\href{http://learnche.mcmaster.ca/feedback-questions}{http://learnche.mcmaster.ca/feedback-questions}
	\vspace{12pt}
	\hrule
	\begin{center}
		\includegraphics[width=0.80\textwidth]{\imagedir/teaching/anonymous-feedback-2015.png}
	\end{center}
\end{frame}

\begin{frame}\frametitle{Main topics covered in 4G3}
	\centerline{\includegraphics[width=0.80\textwidth]{\imagedir/optimization/types-of-optimization-NEOS.png}}
	\begin{enumerate}
		\item	Linear programming (LPs)
		\item	Mixed Integer Linear Programming (MILPs)
		\item	Nonlinear Programming (NLPs) with a single variable (unconstrained)
		\item	Nonlinear Programming with multiple variables 
		\item	Nonlinear Programming with constraints
	\end{enumerate}
\end{frame}

\begin{frame}\frametitle{Linear programming (LP)}
	
	\textbf{Classic examples}: blending (steel example below); diet problem
	
	\vspace{20pt}
	
	\begin{tabular}{|l|ccc|}\hline
		\emph{Component} $\downarrow$	& \textbf{Material 1} & \textbf{Material 2} & \textbf{Material 3} \\ \hline
		Nickel	[\%]		            & 4		              & 7	                & 4                   \\
		Carbon [\%]			            & 0.8	              & 0.7		            & 0.65                \\
		Chromium [\%]		            & 12                  & 0		            & 0	                  \\
		Molybdenum	[\%]	            & 1                   & 2                   & 0                   \\ \hline
		~~~~ Cost			            & \$ 12/kg            & \$ 10/kg            & \$ 6/kg             \\ \hline
		~~~~ Amount available           & 100 kg              & 120 kg              & Unlimited           \\ \hline
		~~~~ Amount used                & $x_1 $              & $x_2$               & $x_3$               \\ \hline
	\end{tabular}
	
	\vspace{12pt}
	
	\textbf{Aim}: produce 500 kg; with minimum cost = 
	
	\textbf{Constraints}?	
\end{frame}

\begin{frame}\frametitle{Mixed integer linear programming (MILP)}
	We have to use either 0 kg or 100 kg of material 1 {\small (``\emph{all or nothing}'')}
	
	We have to use either 0 kg or 120 kg of material 2.
	
	\vspace{12pt}
	How to represent this? Use integer variables.
	
	\vspace{3cm}
	\hrule
	\vspace{2pt}
	
	\emph{Other examples}: 
	
	
	
	\begin{columns}[c]
		\column{0.25\textwidth}
			\begin{itemize}
				\item	budget allocation
				\item	Ontario energy mix
			\end{itemize}
			
		\column{0.75\textwidth}
			\centerline{\includegraphics[width=\textwidth]{\imagedir/optimization/energy-mix-electricity-Marlin.png}}
			
	\end{columns}
\end{frame}

\begin{frame}\frametitle{}
	\centerline{\includegraphics[width=1.15\textwidth]{\imagedir/optimization/supply-chain-explanation-Marlin.png}}
\end{frame}


\begin{frame}\frametitle{Nonlinear programming (NLP) with a single variable}
	There are so many examples of this:
	\begin{enumerate}
		\item	optimal pipe diameter (demonstrated on the board)
		
		\item	selecting the dry-bulb temperature in a dryer
			\begin{itemize}
				\item	higher dry-bulb temperature has an increased cost, 
				\item	but shorter time to dry (so higher throughput possible)
			\end{itemize}
		\item	selecting the throughput (product rate) in a plant, often based on NPV
			\begin{itemize}
				\item	demand
				\item	inventory / storage costs
				\item	equipment size
				\item	costs to produce more scale in a nonlinear way
			\end{itemize}
	\end{enumerate}
\end{frame}

\begin{frame}\frametitle{Nonlinear programming (NLP) with a multiple variables (unconstrained)}
	
	\begin{enumerate}
		\item	Controller tuning:
		\begin{itemize}
			\item	the search variables: $K_C$ and $T_I$
			\item	what could be the objectives here?
		\end{itemize}
		\centerline{\includegraphics[width=.9\textwidth]{\imagedir/process-control/concept-inventory/system-step-responses-no-letters.png}}
		
		\item	Linear regression:
			\begin{itemize}
				\item	what are the search variables?
				\item	what could be the objectives here?
			\end{itemize}
	\end{enumerate}	
\end{frame}

\begin{frame}\frametitle{Nonlinear programming (NLP) with a multiple variables (constrained)}
	
	Where do you locate your business?
	
	\centerline{\includegraphics[width=\textwidth]{\imagedir/optimization/location-objective-function-Benoit.png}}
	
\end{frame}


\begin{frame}\frametitle{What can I do with these topics learned?}
	
	\begin{enumerate}
		\item	Job posting in Abu Dhabi, The Petroleum Institute
		\begin{itemize}
			\item	``Candidates ... have expertise in mathematical programming (mixed-integer, non-convex and stochastic
	programming) and process simulation.''
			
			\item	``The successful candidate will develop models, theory, and algorithms for large scale optimization problem in the fields of the oil, gas, and petrochemical industries as well as in the area of supply chain management.''
		\end{itemize}
		
		
		\item	An article in \href{http://www.forbes.com/sites/susanadams/2014/11/12/top-degrees-for-getting-hired-in-2015/}{Forbes}, November 2014 
	
			\centerline{\includegraphics[width=.6\textwidth]{\imagedir/optimization/forbes-top-bachelor-s-degrees-in-demand-no-of-respondents-that-will-hire_chartbuilder11.png}}
	\end{enumerate}
\end{frame}


\begin{frame}\frametitle{Grading}
	\begin{tabular}{lc}
		Group-submitted assignment/tutorials 					& \textbf{12\%} \\
		Group project (meeting and final reports)				& \textbf{20\%} \\
		Midterm test											& \textbf{20\%} \\
		Final exam												& \textbf{48\%} \\
	\end{tabular}

	\vspace{24pt}
	\begin{itemize}
		\item	There are other minor notes in the course outline: \emph{{\color{myOrange}you must please read it carefully!}}
	\end{itemize}
\end{frame}

\begin{frame}\frametitle{Midterms and exam}

	\begin{itemize}
		\item	Midterm: 25 February, Wednesday evening, starting at 19:00
		
		\vspace{12pt}
		\item	Final exam: cumulative of everything in the course
	\end{itemize}

	\vspace{24pt}
	All tests and exams:
	\begin{itemize}
		\item	closed notes -- no form of paper
		\item	a page of formulas will be provided to you before the midterm and before the exam
		\item	McMaster calculator only
	\end{itemize}
\end{frame}

\begin{frame}\frametitle{Tutorials}
	Tutorials are on Monday and Friday, starting on 09 January
	
	\vspace{12pt}
	Slot \textbf{F} (\emph{Friday}, 10:30-12:20) and \textbf{M} (\emph{Monday}, 15:30-17:20) 

	\begin{itemize}
		\item	Tutorials are in BSB, room 249
		\item	Please bring a laptop, if you have one
	\end{itemize}
	
	\pause
	\begin{exampleblock}{}
		\centering
		{\color{red}Tutorials are \textbf{not optional};} you must show up on time, and stay for the full duration
	\end{exampleblock}
\end{frame}

\begin{frame}\frametitle{Project}
	A major component of the course is the project.

	\vspace{12pt}
	The project gives you a chance to apply the optimization tools to actual problems that you will formulate during the term.

	\vspace{12pt}
	There will be group meetings to judge your progress.
\end{frame}

\begin{frame}\frametitle{Group work}
	\begin{itemize}
		\item	All assignments and projects can be completed in groups
		\item	Groups are 2 people (no exceptions)
	\end{itemize}
\end{frame}

